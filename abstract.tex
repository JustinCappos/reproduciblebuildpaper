\begin{abstract}

Reproducible builds have gained additional deployment and scrutiny in 
recent years.  Many major software projects, including Tor, Debian, Arch
Linux, ... have undertaken massive efforts to make many of their packages
build securely.  As a result, Tor and XXX build reproducibly and 
between 3X-9X\% of packages now build reproducibly for Debian, Arch, etc...
This has been touted as a major step toward improving the security of these
projects.

This paper describes a unexpected benefit of reproducible builds --- discovery
of a wide array of previously unknown bugs.  We propose a slightly unorthodox
philosophy \cappos{Am I overstating this?} specifically dictating \emph{where} 
reproducible build bugs should be fixed rather than just focusing on the goal 
of making builds reproducible.  We report on our experience working on a major
Linux distribution, making XXK projects reproducible (9X\%) and outline the 
effective tools and techniques to do so.  
%Armed with knowledge about the bugs 
%and techniques, \cappos{optional, if we need more `meat'} we built \sysname, 
%a tool that varies the environment in an attempt to trigger bugs that would be 
%found in the reproducible build process.  
In addition to making a major Linux distribution reproduce 9X\% of its 
packages, our philosophy also uncovered XXX unrelated bugs across XX projects, 
including XX critical security flaws.  


\end{abstract}
