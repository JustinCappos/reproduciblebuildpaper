\section{Bugs Found}
\label{SEC:bugs}

This section decribes the flaws found via making software reproducible.  These
bugs are \emph{not} merely situations where reproducibility does not work, but
instead covers (seemingly) unrelated bugs uncovered via this process.

\cappos{likely have a table showing some quantitative data about the bugs.
Hopefully we have at least a few dozen examples.  It would be fine to
categorize and group them it there are too many to list.}

\cappos{General ideas about what was found.  Discussing interesting details of 
most types of bugs.}

{\bf race conditions.}
One type of bug that was frequently found were race conditions in the build
process.  This was often the case where a build script would start multiple
copies of the same tool.  One cause for this was when those programs would 
share common temp files, caches, etc. \cappos{cite
\url{https://bugzilla.opensuse.org/show_bug.cgi?id=1021353}
\url{https://bugs.launchpad.net/intltool/+bug/1687644} }.  The issue is that the
programs will either overwrite or corrupt each other's state, leading to
errors in the output files.  \cappos{What issues does this cause in the
installed software?  Is there some sort of security / stability flaw in the
eventual code?}
In these two cases, this only affected manual pages, but it is possible that 
similar flaws exist in code that produces executable output.

Another way that race conditions cause issues deals with checking
timestamps on files.  For example, a build script may start two different
programs that check some property of timestamps and decide how to behave.
For example, in the build script for {\tt python-bottle}, the Python2 and 
Python3 build scripts are run one after another.  However, if both the
Python2 and Python3 build scripts complete within the same second the
2nd install script will think it is already done, because the output file
size and timestamp already have the expected value.

\cappos{are there time-of-check-to-time-of-use bug or something else?}

Another variant on this involves copying build files over each other at
different times.  For example, in {\tt mtr} the {\tt Makefile.dist} file 
was being copied over the {\tt Makefile}, leading to a race condition.
To quote the bug report:
\cappos{cite
\url{http://pkgs.fedoraproject.org/cgit/rpms/mtr.git/commit/?id=9dd4325251e1c28064f4bf5b84ae2f95e3118200}}
Note that make doesn't wait for this background task to finish. During
rpm build we are building mtr twice. After first build we call
distclean. If second invocation of configure script runs in less than 3
seconds then the Makefile generated by configure will be overwritten by
background copy. We don't want that and since we are calling configure
explicitly we don't really need this "feature" at all.
