\section{Reproducible Builds}
\label{SEC:reproducible-builds}

\cappos{1 para history / motivation.  Set up the next sections}

\subsection{Reproducible builds}

According to (\cappos{cite
https://reproducible-builds.org/docs/definition/}:
A build is reproducible if given the same source code, build environment
and build instructions, any party can recreate bit-by-bit identical copies
of all specified artifacts.

The relevant attributes of the build environment, the build instructions
and the source code as well as the expected reproducible artifacts are
defined by the authors or distributors. The artifacts of a build are the
parts of the build results that are the desired primary output.


Following that terminology, we also define:

{\bf Source code} is usually a checkout from version control at a specific
revision or a source code archive.

{\bf Relevant attributes} of the build environment would usually include
dependencies and their versions, build configuration flags and environment
variables as far as they are used by the build system (eg. the locale). It
is preferable to reduce this set of attributes.

{\bf Artifacts} would include executables, distribution packages or filesystem
images. They would not usually include build logs or similar ancillary
outputs.

The reproducibility of artifacts is {\bf verified} by bit-by-bit comparison. 
This is usually performed using cryptographically secure hash functions.

{\bf Authors or distributors} means parties that claim reproducibility of a 
set of artifacts. These may be upstream authors, distribution maintainers or
any other distributor.

\cappos{text to consider integrating.  Entropy is a loaded term in academia, 
so we would need to either be very careful with how we use it or rephrase it.}
A "build process" is
a complicated function that has a complicated output.  Ways to make that
function deterministic include (1) fixing the input to the function
(VM), (2) using a comparator that ignores some of the output bits
(strip-nondeterminism), (3) changing the function so it is
deterministic.  Intuitively, since (3) dives into the guts of the build
process (in contrast to (1) and (2) which are more black-boxy), it is
the most difficult/invasive, but also the most <handwave>correct</handwave>
and the most rewarding.

\subsection{What should be assumed about the build environment?}

\cappos{3-4 paragraphs, itemized list}

