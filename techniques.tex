\section{Reproducible Build Techniques}
\label{SEC:techniques}

\cappos{describe all the tools / techniques here.
SOURCE\_DATE\_EPOCH, strip-nondeterminism, docker containers, reprotest, etc.
2-4 paragraphs per tool / technique}

\begin{itemize}

\item {\tt SOURCE\_DATE\_EPOCH} is a distribution-agnostic standard for
upstream build processes to consume a determinstic timestamp from packaging
systems.
Packaging systems set the environment variable \texttt{SOURCE\_DATE\_EPOCH} to the timestamp of the original public release of the source code being built,
and build processes consume that value and use it instead of the system timestamp where appropriate,
such as when embedding timestamps into user-visible portions of generated documentation.
This allows a particular source tree \cappos{DS: need to define a term for `source package/archive'---meaning a `.tar.gz' or a tag---where `Artifact' is defined} to remain reproducible even if built on different dates.
\cappos{how does
this deal with elapsed time?}

The value of {\tt SOURCE\_DATE\_EPOCH} is a UNIX timestamp, defined as the number of seconds
(excluding leap seconds) since 01 Jan 1970 00:00:00 UTC, and is exported 
through the operating system's usual environment mechanism.
On C-based system, it is available via the {\tt getenv(3)} standard library function.
\url{https://reproducible-builds.org/specs/source-date-epoch/}
?Dozens? of popular build tools, including XX, YY, and ZZ, have been 
modified to use {\tt SOURCE\_DATE\_EPOCH}.

The use of {\tt SOURCE\_DATE\_EPOCH} is a compromise solution that is not 
universally loved by reproducible builds projects for a few reasons
(cite https://wiki.debian.org/ReproducibleBuilds/StandardEnvironmentVariables 
):
First, it requires changes in the build system for different projects, which
makes it difficult to gain adoption in some cases.  Instead it may be easier
to simply have the build system omit information that leads to non-determinism.
Second, this technique breaks build systems, because files will have the
same timestamp (e.g., setting the system clock to some constant time T, 
breaks various build tools such as GNU make).  Third, it also does not
actually remove the irreproducibility issues.  Setting the system clock to 
some constant time T at the start of the build, then letting it run normally, 
does not achieve reproducibility in the cases where the build takes a 
non-deterministic amount of time and something embeds ``the current time" 
near the end of the build process.  Finally, mandating a specific start
time or build path prevents parallel builds from being done without extra
system-level support (e.g. virtualisation or separate kernel namespaces),
and prevents users without permissions to use that path from verifying
reproducibility.  Combined, these limitations make {\tt
SOURCE\_DATE\_EPOCH} a solution with significant drawbacks.

\cappos{DS: something about `WW packages had build processes that would fail within their support lifetime, and have been fixed'---this makes developing/backporting vulnerability patches easier}



\item {\em strip-nondeterminism} is a tool that is run over the finished
archives and image files that removes common sources of non-determinism.
This includes timestamps and similar metadata from the files.  \cappos{need
much more detail.}  The drawback of this is...

\end{itemize}

In addition, there are several tools that are used to aid efforts to make
builds reproducible.

\begin{itemize}

\item {\em diffoscope} is a recursive and content-aware diff utility used
to locate and diagnose reproducibility issues. Unlike the traditional {\tt
diff(1)} utility which mostly focuses on determining if files differ, 
{\em diffoscope} is optimized to help dig through differences that relate 
to reproducible builds.  One major improvement is in the way that binary
differences are handled.  Rather than just observing a difference,
diffoscope parses the files in the appropriate format and understands the
context of the change.  For example, differences in PDFs fields may be
decoded to indicate that the difference is in the timestamp field.
\cappos{fix?}

Another major improvement is that diffoscope will recursively unpack 
archives of many kinds.  Package formats are usually just standard archive
formats, with additional fields that indicate metadata for the package.  In
fact, many archive formats contain several archives, which can be nested
inside each other.
For example, the Hello World Debian package \texttt{hello\_2.10-1+b1\_amd64.deb} has an outer AR archive,
which contains
	a format file,
	a gzip-compressed tarball containing metadata,
	and an xzip-compressed tarball containing the files to be installed---%
	some of which are themselves compressed.
In larger packages, the files to be installed may include archive files, such as JAR or ISO files.
By
recursively processing files, diffoscope can pinpoint that a difference is
due to the XX inside of YY inside of ZZ in this package.

\item {\em a .buildinfo file} captures information about the system
environment in which a build is performed.  One obvious use of this
information is to examine the environment a build happened in, perhaps for
forensic purposes.  However, a {\tt .buildinfo} file may also be used to 
recreate that build environment later.  This enables different users to
build software in a reproducible manner, with the {\tt .buildinfo} as a
guide.  \cappos{How do we clearly distinguish this from requiring something
like {\tt SOURCE\_DATE\_EPOCH}?}

It is also envisioned that the {\tt .buildinfo} file may have a variety of
other uses in the future.  First, a debian user could elect to only install 
binary packages that have been successfully built by multiple builders.
\cappos{How does this work?  Are these files signed?}   Second, a {\tt
.buildinfo} file could be used by a debian derivative to attest to the 
packages it has deterministically rebuilt.  Third, a triage process could
use {\tt .buildinfo} files to identify toolchain changes that have an effect on
large numbers of binary packages (e.g., a backdoor in a commonly used
library cite Xcode Ghost).  Finally, {\tt .buildinfo} files used as part 
of a cross-building process could demonstrate the correctness of the
cross-compiling toolchain by reproducing the exact binary artifacts.

\end{itemize}
