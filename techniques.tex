\section{Reproducible Build Techniques}
\label{SEC:techniques}

\cappos{describe all the tools / techniques here.
SOURCE\_DATE\_EPOCH, strip-nondeterminism, docker containers, reprotest, etc.
2-4 paragraphs per tool / technique}

\begin{itemize}

\item {\tt SOURCE\_DATE\_EPOCH} is a distribution-agnostic standard for
upstream build processes to consume a determinstic timestamp from packaging
systems.
Packaging systems set the environment variable \texttt{SOURCE\_DATE\_EPOCH} to the timestamp of the original public release of the source code being built,
and build processes consume that value and use it instead of the system timestamp where appropriate,
such as when embedding timestamps into user-visible portions of generated documentation.
This allows a particular source tree \cappos{DS: need to define a term for `source package/archive'---meaning a `.tar.gz' or a tag---where `Artifact' is defined} to remain reproducible even if built on different dates.
\cappos{how does
this deal with elapsed time?}

The value of {\tt SOURCE\_DATE\_EPOCH} is a UNIX timestamp, defined as the number of seconds
(excluding leap seconds) since 01 Jan 1970 00:00:00 UTC, and is exported 
through the operating system's usual environment mechanism.
On C-based system, it is available via the {\tt getenv(3)} standard library function.
\url{https://reproducible-builds.org/specs/source-date-epoch/}
?Dozens? of popular build tools, including XX, YY, and ZZ, have been 
modified to use {\tt SOURCE\_DATE\_EPOCH}.

\cappos{DS: something about `WW packages had build processes that would fail within their support lifetime, and have been fixed'---this makes developing/backporting vulnerability patches easier}

\item {\em diffoscope} is a recursive and content-aware diff utility used
to locate and diagnose reproducibility issues. Unlike the traditional {\tt
diff(1)} utility which mostly focuses on determining if files differ, 
{\em diffoscope} is optimized to help dig through differences that relate 
to reproducible builds.  One major improvement is in the way that binary
differences are handled.  Rather than just observing a difference,
diffoscope parses the files in the appropriate format and understands the
context of the change.  For example, differences in PDFs fields may be
decoded to indicate that the difference is in the timestamp field.
\cappos{fix?}

Another major improvement is that diffoscope will recursively unpack 
archives of many kinds.  Package formats are usually just standard archive
formats, with additional fields that indicate metadata for the package.  In
fact, many archive formats contain several archives, which can be nested
inside each other.
For example, the Hello World Debian package \texttt{hello\_2.10-1+b1\_amd64.deb} has an outer AR archive,
which contains
	a format file,
	a gzip-compressed tarball containing metadata,
	and an xzip-compressed tarball containing the files to be installed---%
	some of which are themselves compressed.
In larger packages, the files to be installed may include archive files, such as JAR or ISO files.
By
recursively processing files, diffoscope can pinpoint that a difference is
due to the XX inside of YY inside of ZZ in this package.

\end{itemize}
